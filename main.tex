%%%%%%%%%%%%%%%%%%%%%%%%%%%%%%%%%%%%%%%%%
% Conference Booklet
% LaTeX Template
% Version 1.0 (22/12/2019)
%
% This template originates from:
% https://www.LaTeXTemplates.com
%
% Authors:
% Maxime Lucas (ml.maximelucas@gmail.com)
% Pau Clusella
% Modifications for LaTeX Templates by Vel (vel@LaTeXTemplates.com)
%
% License:
% GNU General Public License v3.0
%
%%%%%%%%%%%%%%%%%%%%%%%%%%%%%%%%%%%%%%%%%

%----------------------------------------------------------------------------------------
%	PACKAGES AND OTHER DOCUMENT CONFIGURATIONS
%----------------------------------------------------------------------------------------

\documentclass[
	openany, % Allow chapters to start on odd and even pages
	parskip=full, % Large space between paragraphs
	12pt, % Default font size
	a4paper, % Paper size, use letterpaper for US letter size
]{conferencebooklet} % Custom class defining the style and layout of the template

%----------------------------------------------------------------------------------------

\begin{document}

%----------------------------------------------------------------------------------------
%	 COVER PAGE
%----------------------------------------------------------------------------------------

%\includepdf{images/cover.jpg} % The cover for the booklet is included as a whole-page image, it can be a PDF or an image file but must be the same dimensions as the paper size

%----------------------------------------------------------------------------------------
%	 INFORMATION/COPYRIGHT PAGE
%----------------------------------------------------------------------------------------

%\thispagestyle{empty} % Suppress headers and footers on this page

%~\vfill % Push text down

% \begin{center}
% 	This is the short version of the booklet for print use. \\ Full abstracts with all authors, references, and figures can be found at:\\ \url{https://amcosconference.com/}

% 	This template originates from \url{LaTeXTemplates.com} and is based on the original version at:\\ \url{https://github.com/maximelucas/AMCOS\_booklet}
% \end{center}


\vspace*{3cm}
\begin{center}

%Program and Abstracts
%
%\vspace*{5mm}
%
%\hrulefill \Huge


\thispagestyle{empty}
\LARGE\textbf{CAS AMSS-PolyU Joint Laboratory of Applied Mathematics Workshop 2022}



%\hrulefill

\vspace*{2cm}
\Large

December 22--23, 2022 (Zoom)

\end{center}

%\newpage
%\thispagestyle{empty}
\vspace*{10cm}


\noindent
%{\it Sponsors:} \\[2ex]
%AMSS-PolyU Joint Research Institute\\
%Hong Kong Mathematical Society,\\
%Department of Applied Mathematics, PolyU\\



% \begin{tabular}{lll}
% {\textbf{Sponsors:}}  & &\textbf{Enquiry:} \\ [2ex]
% AMSS-PolyU Joint Research Institute & &Dr. Zhonghua Qiao\\
% Hong Kong Mathematical Society & &Tel: 2766 6932\\
% Department of Applied Mathematics, PolyU & &Email: zqiao@inet.polyu.edu.hk\\
% \end{tabular}


%\addtocounter{page}{-1}
%----------------------------------------------------------------------------------------
%	 TABLE OF CONTENTS
%----------------------------------------------------------------------------------------
\setcounter{page}{0}
\tableofcontents

%----------------------------------------------------------------------------------------
%	 ABOUT CONFERENCE
%----------------------------------------------------------------------------------------

\chapter{About}



\section{CAS AMSS-PolyU Joint Laboratory of Applied Mathematics Workshop}

This workshop is organized by CAS AMSS-PolyU  Joint Laboratory of Applied Mathematics\footnote{The AMSS-PolyU Joint Research Institute (JRI) for Engineering and Management Mathematics was jointly established by the AMSS and PolyU in 2005. Approved by CAS, the JRI has upgraded to one of 22 CAS joint laboratories with Hong Kong universities in 2019 and named CAS AMSS-PolyU Joint Laboratory of Applied Mathematics (JLab).\\
JLab website: \url{https://www.polyu.edu.hk/ama/research-and-consultancy/cas-amss-polyu-jlab/}\,.\\ \\ } and supported by AMSS, PolyU, AMA and CAS-Croucher Funding Scheme for Joint Laboratories.



\begin{center}
	\begin{itemize}
	    \item[*] CAS: Chinese Academy of Sciences
	    \item[*] AMSS: Academy of Mathematics and Systems Science
	    \item[*] PolyU: The Hong Kong Polytechnic University
	    \item[*] AMA: Department of Applied Mathematics, PolyU
	\end{itemize}
\end{center}


%----------------------------------------------------------------------------------------
%	 TIMETABLE
%----------------------------------------------------------------------------------------

\chapter{Program}

\section{Thursday, 22 December}

% \renewcommand*{\arraystretch}{2.2}

% \begin{center}
%     \begin{table}[htbp]
%     \begin{tabular}{|cll|}\hline
%     \tablebreak{09:00--09:15}{Welcome Remarks\quad Chair: Defeng Sun}\\
%     & \makecell[l]{Ya-xiang Yuan\\ AMSS} & Speech \\
%     \hline
% 	& \makecell[l]{Xiaojun Chen\\ PolyU}  & Speech \\
% 	\hline
% 	\tablebreak{09:15--11:45}{Session 1: Differential Equations\quad Chair: Zhonghua Qiao}\\
% 	09:15 & \makecell[l]{Fanghua Lin\\ New York University} & \color{red}{TBA} \\
%     \hline
%     10:00  & Coffee Break  &  \\
% 	\hline
% 	10:15 & \makecell[l]{Ping Zhang\\ AMSS}  & \makecell[l]{Gevrey solutions of quasi-linear hyperbolic\\ hydrostatic Navier-Stokes~system}\\
% 	\hline
% 	11:00 & \makecell[l]{Tong Yang\\ PolyU}  & \color{red}{TBA} \\
%     \hline
%     \tablebreak{14:30--17:00}{Session 2: Mathematical Finance\quad Chair: Zhan Shi}
%     14:30 & \makecell[l]{Min Dai\\ PolyU} & \makecell[l]{Strategic Investment under Uncertainty with\\ First-and Second-mover Advantages} \\
%     \hline
%     15:15 & \makecell[l]{Yongsheng Song\\ AMSS} & \makecell[l]{The Central Limit Theorem and the Law of\\ Large Numbers under Sublinear Expectations} \\
%     \hline
%     16:00  & Coffee Break  &  \\
% 	\hline
%     16:15 & \makecell[l]{Nizar Touzi\\ Ecole Polytechique} & \makecell[l]{Mean field game of mutual holding and\\ systemic risk} \\
%     \hline
%     \end{tabular}
% \end{table}
% \end{center}




\renewcommand*{\arraystretch}{2.2}
\begin{longtable}{|C{0.15\linewidth} L{0.25\linewidth} L{0.46\linewidth}|}\hline
	\tablebreak{09:00--09:15}{Welcome Remarks\quad Chair: Defeng Sun (PolyU)}\\
	& \makecell[l]{Ya-xiang Yuan\\ AMSS} & Speech \\
    \hline
	& \makecell[l]{Xiaojun Chen\\ PolyU}  & Speech \\
	\hline
	\tablebreak{09:15--11:45}{Session 1: Differential Equations\quad Chair: Zhonghua Qiao (PolyU)}\\
	09:15 & \makecell[l]{Fanghua Lin\\ New York University} & \color{red}{TBA} \\
    \hline
    10:00  & Coffee Break  &  \\
	\hline
	10:15 & \makecell[l]{Ping Zhang\\ AMSS}  & \makecell[l]{Gevrey Solutions of Quasi-linear Hyperbolic\\ Hydrostatic Navier-Stokes~System}\\
	\hline
	11:00 & \makecell[l]{Tong Yang\\ PolyU}  & \makecell[l]{Analysis on Tollmien-Schlichting Waves in\\ MHD and Compressible Fluid} \\
    \hline
    \tablebreak{14:30--17:00}{Session 2: Mathematical Finance\quad Chair: Zhan Shi (AMSS)}
    14:30 & \makecell[l]{Min Dai\\ PolyU} & \makecell[l]{Strategic Investment under Uncertainty with\\ First-and Second-mover Advantages} \\
    \hline
    15:15 & \makecell[l]{Yongsheng Song\\ AMSS} & \makecell[l]{The Central Limit Theorem and the Law of\\ Large Numbers under Sublinear Expectations} \\
    \hline
    16:00  & Coffee Break  &  \\
	\hline
    16:15 & \makecell[l]{Nizar Touzi\\ Ecole Polytechique} & \makecell[l]{Mean Field Game of Mutual Holding and\\ Systemic Risk} \\
    \hline


% 	\Lec{16:15--17:00}{Nizar Touzi}{Ecole Polytechique}{Mean field game of mutual holding and systemic risk}
% 	%\eventtype{17:10--19:30}{Poster session with Wine \& Cheese}
\end{longtable}

\newpage

%------------------------------------------------
\chapter*{Program}

\section{Friday, 23 December}

\begin{longtable}{|C{0.15\linewidth} L{0.25\linewidth} L{0.46\linewidth}|}\hline
	\tablebreak{09:15--11:45}{Session 3: Statistics\quad Chair: Xingqiu Zhao (PolyU)}\\
	09:15 & \makecell[l]{Tengyuan Liang\\ University of Chicago} & Universal Prediction Band and Variance Interpolation via Semi-Definite Programming \\
    \hline
    10:00  & Coffee Break  &  \\
	\hline
	10:15 & \makecell[l]{Jian Huang\\ PolyU}  & Conditional Deep Generative Learning \\
	\hline
	11:00 & \makecell[l]{Xinyu Zhang\\ AMSS}  & Optimal Parameter-transfer Learning by Semiparametric Model Averaging \\
    \hline
    \tablebreak{14:30--17:00}{Session 4: Optimization\quad Chair: Yu-Hong Dai (AMSS)}\\
    14:30 & \makecell[l]{Xin Liu\\ AMSS} & Optimization Models and Approaches for Strictly Correlated Electrons \\
    \hline
    15:15 & \makecell[l]{Houduo Qi\\ PolyU} & Euclidean Distance Matrix Optimization and Its Application to Portfolio Theory \\
    \hline
    16:00  & Coffee Break  &  \\
	\hline
    16:15 & \makecell[l]{Daniel Kuhn\\ Ecole Polytechique\\ Federale de Lausanne} & A General Framework for Optimal Data-Driven Optimization \\
    \hline
    \tablebreak{17:00--17:15}{Closing}

% 	\eventtype{20:00}{Conference Dinner}
\end{longtable}

%------------------------------------------------

% \section{Thursday, 22 of March}

% \begin{longtable}{|C{0.15\linewidth}| C{0.04\linewidth}|  C{0.3\linewidth} C{0.0\linewidth} C{0.4\linewidth}|}\hline
% 	\IS{9:00 -- 9:40}{Hiroya Sato}{Tokyo, Japan}{Title of invited speaker}
% 	\IS{9:40--10:20}{Hiroya Sato}{Tokyo, Japan}{Title of invited speaker}
% 	\IT{10:20--10:45}{Franck Schmidt}{Munich, Germany}{A special talk about diversity in science}
% 	\tablebreak{10:45--11:10}{Coffee}
% 	\CT{11:10-11:40}{Marc Jansen}{Amsterdam, The Netherlands}{Title of contributed talk and references and a figure}
% 	\KL{11:40--12:35}{Leon Tremblay}{Montreal, Canada}{Title of a keynote lecture}
% 	\eventtype{12:35--12:45}{Poster Prize \& Conclusion}
% 	\tablebreak{12:45--14:00}{Lunch}
% \end{longtable}

%----------------------------------------------------------------------------------------
%	 LIST OF TALK ABSTRACTS
%----------------------------------------------------------------------------------------

% Abstract template
%\abstract
%	{} % Title
%	{} % Author(s)
%	{} % Tag, can be: empty, \KLtag (keynote lecture), \IStag (invited speaker), \CTtag (contributed talk) or \ITtag (invited talk)
%	{} % Affiliation(s)
%	{} % Abstract text

\chapter{Abstracts}

\abstract
    {\textcolor{red}{TBA}}
    {Fanghua  Lin}
    {}
    {New York University\\
    Email: linf@cims.nyu.edu}
    {\textcolor{red}{TBA}}

\chapter*{Abstracts}
\abstract
    {Gevrey Solutions of Quasi-linear Hyperbolic Hydrostatic Navier-Stokes System}
    {Ping Zhang}
    {}
    {AMSS, Chinese Academy of Sciences\\
    Email: zp@amss.ac.cn}
    {We study the well-posedness of  a hyperbolic quasilinear version of hydrostatic  Navier-Stokes system in $\mathbb{R}\times \mathbb{T}$, and  prove the global well-posedness of the system with initial  data which are small and analytic  in both variables. We also prove  the convergence of such analytic solutions to that of the classical hydrostatic Navier-Stokes system when the delay time converges to zero. Furthermore, we obtain a local well-posedness result in Gevrey class $2$ when the initial datum is a small perturbation of some convex function.}


\chapter*{Abstracts}
\abstract
    {Analysis on Tollmien-Schlichting Waves in MHD and Compressible Fluid}
    {Tong Yang}
    {}
    {The Hong Kong Polytechnic University\\
    Email: t.yang@polyu.edu.hk}
    {In a region closer to the boundary compared to Prandtl layer, an inviscid disturbance can be manifested by the interaction with viscous mode via the no-slip boundary condition due to resonance. In some unstable range of parameters, this leads to instability in the transition regime from laminar flow to turbulence. This instability phenomenon was observed by physicists long time ago, such as Heisenberg, Tollmien and  C.C. Lin, etc.  And it was justified rigorously in mathematics by Grenier-Guo-Nguyen using the incompressible Navier-Stokes equation. In this talk, we will present some results on this phenomenon in other physical situations that the governing system is either MHD or compressible Navier-Stokes equation.  The talk is based on some recent joint work with Chengjie Liu and Zhu Zhang.}


\chapter*{Abstracts}
\abstract
    {Strategic Investment under Uncertainty with First-and Second-mover Advantages}
    {Min Dai}
    {}
    {The Hong Kong Polytechnic University\\
    Email: minpolyuhk.dai@polyu.edu.hk}
    {We analyze a duopoly entry game where firms trade off the first-mover advantage (of earning monopoly rents) against the second-mover advantage (of paying a lower entry cost) in the classic real-option framework. We show that the equilibrium value function is governed by a variational inequality. The equilibrium solution features five regions. There are two waiting regions due to two distinct waiting motives: a new waiting-to-save-entry-costs and the standard option-value-of-waiting motives. For sufficiently high market demand, there is no first-mover advantage in equilibrium as Follower immediately enters after Leader. Therefore, firms use mixed strategies to enter as Leader with a rate increasing in market demand, giving rise to a probabilistic entry region. For intermediate levels of market demand, firms rush to enter in the first-mover-advantage-induced “rent-equalization” region (Fudenberg and Tirole, 1985; Grenadier, 1996). Finally, a second probabilistic entry region (where Follower waits so that Leader collects some monopoly rents) emerges to connect the rent-equalization region and the waiting-to-be-the-second-mover region. Quantitatively, the second-mover advantage can cause firms to significantly delay entry and substantially erode firm value. This work is joint with Zhaoli Jiang and Neng Wang.}


\chapter*{Abstracts}
\abstract
    {The Central Limit Theorem and the Law of Large Numbers
    under Sublinear Expectations}
    {Yongsheng Song}
    {}
    {AMSS, Chinese Academy of Sciences\\
    Email: yssong@amss.ac.cn}
    {We first introduce Stein's method under sublinear expectations, by which we give convergence rates for the central limit theorem and the (weak) law of large numbers under sublinear expectations (CLT* \& LLN*). Then we give a version of strong LLN*  as the sublinear expectation defined on a Polish space is regular, which shows that any constant $\mu$ in the mean interval $[\underline{\mu},\overline{\mu}]$ can be considered as a limit of the emperical averages.}


\chapter*{Abstracts}
\abstract
    {Mean Field Game of Mutual Holding and Systemic Risk}
    {Nizar Touzi}
    {}
    {Ecole Polytechnique\\
    Email: nizar.touzi@polytechnique.edu}
    {We provide an explicit solution for the mean-field game of mutual holding with defaultable agents modeled by absorption at zero. The optimal dynamics are defined by a Mckean-Vlasov SDE with a discontinuous diffusion coefficient and nonsmooth drift coefficient. We also provide an autonomous characterization of the distribution of defaults.}

%------------------------------------------------


\chapter*{Abstracts}
\abstract
    {Universal Prediction Band and Variance Interpolation via Semi-Definite Programming}
    {Tengyuan Liang}
    {}
    {University of Chicago\\
    Emails: Tengyuan.Liang@chicagobooth.edu}
    {We propose a computationally efficient method to construct nonparametric, heteroscedastic prediction bands for uncertainty quantification, with or without any user-specified predictive model. Our approach provides an alternative to the now-standard conformal prediction for uncertainty quantification, with novel theoretical insights and computational advantages. The data-adaptive prediction band is universally applicable with minimal distributional assumptions, has strong non-asymptotic coverage properties, and is easy to implement using standard convex programs. Our approach can be viewed as a novel variance interpolation with confidence and further leverages techniques from semi-definite programming and sum-of-squares optimization. Theoretical and numerical performances for the proposed approach for uncertainty quantification are analyzed.}



\chapter*{Abstracts}
\abstract
    {Conditional Deep Generative Learning}
    {Jian Huang}
    {}
    {The Hong Kong Polytechnic University\\
    Email: j.huang@polyu.edu.hk}
    {Conditional distribution is a fundamental quantity in statistics and machine learning that provides a full description of the relationship between a response and a predictor. There is a vast literature on conditional density estimation. A common feature of the existing methods is that they seek to estimate the functional form of the conditional density. We propose a deep generative approach to learning a conditional distribution by estimating a conditional generator, so that a random sample from the target conditional distribution can be obtained by transforming a sample from a simple reference distribution. The conditional generator is estimated nonparametrically with neural networks by matching appropriate joint distributions using a discrepancy measure. There are several advantages of the proposed generative approach over the classical methods for conditional density estimation, including: (a) there is no restriction on the dimensionality of the response or predictor, (b) it can handle both continuous and discrete type predictors and responses, and (c) it is easy to obtain estimates of the summary measures of the underlying conditional distribution by Monte Carlo. We show that the proposed conditional learning approach can mitigate the curse of dimensionality under a low-dimensional data support assumption. We also conduct extensive numerical experiments to validate the proposed method and using several benchmark datasets, including the California housing, the MNIST, and the CelebA datasets, to illustrate its applications in conditional sample generation, uncertainty quantification of prediction, visualization of multivariate data, image generation and image reconstruction.}



\chapter*{Abstracts}
\abstract
    {Optimal Parameter-transfer Learning by Semiparametric Model Averaging}
    {Xinyu Zhang}
    {}
    {AMSS, Chinese Academy of Sciences\\
    Email: xinyu@amss.ac.cn}
    {In this article, we focus on prediction of a target model by transferring the information of source models. To be flexible, we use semiparametric additive frameworks for the target and source models. Inheriting the spirit of parameter-transfer learning, we assume that different models possibly share common knowledge across parametric components that is helpful for the target predictive task. Unlike existing parameter-transfer approaches, which need to construct auxiliary source models by parameter similarity with the target model and then adopt a regularization procedure, we propose a frequentist model averaging strategy with a J-fold cross-validation criterion so that auxiliary parameter information from different models can be adaptively utilized through data-driven weight assignments. The asymptotic optimality and weight convergence of our proposed method are built under some regularity conditions. Extensive numerical results demonstrate the superiority of the proposed method over competitive methods.}



\chapter*{Abstracts}
\abstract
    {Optimization Models and Approaches for Strictly Correlated Electrons}
    {Xin Liu}
    {}
    {AMSS, Chinese Academy of Sciences\\
    Email: liuxin@lsec.cc.ac.cn}
    {In electronic structure calculations, Kohn-Sham equations rank among the most widely adopted mathematical models. However, due to the deficiency of available approximations for exchange-correlation energy, Kohn-Sham equations cannot well describe strictly correlated electrons at present. To this end, some models based on the strong-interaction limit of density functional theory have been developed in recent decades. The associated energy minimizations can be formulated as multi-marginal optimal transport problems with Coulomb cost (MMOT). Since the curse of dimensionality resides in MMOT, its low-dimensional reformulations are indispensable. In this talk, we consider the reformulation based on a Monge-like ansatz. We discuss the difficulties in the corresponding optimization problems, and also propose a global optimization approach for numerical resolution.}




\chapter*{Abstracts}
\abstract
    {Euclidean Distance Matrix Optimization and Its Application to Portfolio Theory}
    {Hou-Duo Qi}
    {}
    {The Hong Kong Polytechnic University\\
    Email: H.Qi@soton.ac.uk}
    {Euclidean distance matrix (EDM) optimization has proven very useful in data embedding in Euclidean space. This talk reviews some fundamental geometric embedding theory and illustrates its application to portfolio construction. In particular, we study the implication of EDM optimization to the maximum diversification return portfolio, proposed by Booth and Fama in 1990s. We are able to develop diversification return and risk relationship for the efficient frontier and derive the efficient portfolio that has the largest diversification return. Essential to this new development is the concept of portfolio centrality, that is closely related to the principal coordinate system defined by EDM. We will use DAX Index 30 stocks to demonstrates the reported results.}




\chapter*{Abstracts}
\abstract
    {A General Framework for Optimal Data-Driven Optimization}
    {Daniel Kuhn}
    {}
    {Ecole Polytechique Federale de Lausanne\\
    Email: daniel.kuhn@epfl.ch}
    {We propose a statistically optimal approach to construct data-driven decisions for stochastic optimization problems. Fundamentally, a data-driven decision is simply a function that maps the available training data to a feasible action. It can always be expressed as the minimizer of a surrogate optimization model constructed from the data. The quality of a data-driven decision is measured by its out-of-sample risk. An additional quality measure is its out-of-sample disappointment, which we define as the probability that the out-of-sample risk exceeds the optimal value of the surrogate optimization model. The crux of data-driven optimization is that the data-generating probability measure is unknown. An ideal data-driven decision should therefore minimize the out-of-sample risk simultaneously with respect to every conceivable probability measure (and thus in particular with respect to the unknown true measure). Unfortunately, such ideal data-driven decisions are generally unavailable. This prompts us to seek data-driven decisions that minimize the out-of-sample risk subject to an upper bound on the out-of-sample disappointment - again simultaneously with respect to every conceivable probability measure. We prove that such Pareto-dominant data-driven decisions exist under conditions that allow for interesting applications: the unknown data-generating probability measure must belong to a parametric ambiguity set, and the corresponding parameters must admit a sufficient statistic that satisfies a large deviation principle. If these conditions hold, we can further prove that the surrogate optimization model generating the optimal data-driven decision must be a distributionally robust optimization problem constructed from the sufficient statistic and the rate function of its large deviation principle. This shows that the optimal method for mapping data to decisions is, in a rigorous statistical sense, to solve a distributionally robust optimization model. Maybe surprisingly, this result holds irrespective of whether the original stochastic optimization problem is convex or not and holds even when the training data is non-i.i.d. As a byproduct, our analysis reveals how the structural properties of the data-generating stochastic process impact the shape of the ambiguity set underlying the optimal distributionally robust optimization model.}

%----------------------------------------------------------------------------------------
%	 LIST OF POSTERS
%----------------------------------------------------------------------------------------

% \chapter{List of Posters}

% \vspace{-2.5em}

% \section{Tuesday Session}

% \abstract
% 	{A random abstract title} % Title
% 	{\underline{J. Doe}$^{1}$, M. Smith$^{2, 3}$} % Author(s)
% 	{} % Tag, can be: empty, \KLtag (keynote lecture), \IStag (invited speaker), \CTtag (contributed talk) or \ITtag (invited talk)
% 	{$^1$ Faculty of Science and Technology, Bournemouth University, UK\\ $^2$ Institut d'Investigacions Biom\`{e}diques August Pi i Sunyer (IDIBAPS), Barcelona, Spain\\ $^3$ ICREA Barcelona, Spain\\ $^4$ Bernstein Center for Computational Neuroscience, Medical Faculty Mannheim and Heidelberg University, Germany.} % Affiliation(s)
% 	{On recommend tolerably my belonging or am [1]. Mutual has cannot beauty indeed now sussex merely you. It possible no husbands jennings ye offended packages pleasant he. Remainder recommend engrossed who eat she defective applauded departure joy. Get dissimilar not introduced day her apartments. Fully as taste he mr do smile abode every. Luckily offered article led lasting country minutes nor old. Happen people things oh is oppose up parish effect. Law handsome old outweigh humoured far appetite. Now residence dashwoods she excellent you. Shade being under his bed her. Much read on as draw. Blessing for ignorant exercise any yourself unpacked. Pleasant horrible but confined day end marriage. Eagerness furniture set preserved far recommend. Did even but nor are most gave hope. Secure active living depend son repair day ladies now. Affronting discretion as do is announcing. Now months esteem oppose nearer enable too six. She numerous unlocked you perceive speedily. Affixed offence spirits or ye of offices between. Real on shot it were four an as. Absolute bachelor rendered six nay you juvenile. Vanity entire an chatty to [2].

% 	\textbf{References}\newline [1] Sanchez-Vives M.V., Massimini M. and Mattia M. (2017). Neuron, 94(5):993-1001.\newline [2] Capone C., Rebollo B., Mu{\~n}oz A., Illa X., Del Giudice P., Sanchez-Vives M.V., Mattia M. (2017). Cereb Cortex, 28:1-17.} % Abstract text

% %------------------------------------------------

% \section{Alternative Compact Poster List}

% \poster
% 	{A poster title} % Title
% 	{John Smith} % Author(s)
% 	{University Name} % Affiliation(s)

% \poster
% 	{Another poster title} % Title
% 	{Jane Smith} % Author(s)
% 	{University Name} % Affiliation(s)

% %----------------------------------------------------------------------------------------
% %	 LIST OF PARTICIPANTS
% %----------------------------------------------------------------------------------------

% \chapter{List of Participants}

% \begingroup
% 	\rowcolors{1}{gray!25}{white} % Alternating row colours
% 	\begin{longtable}{p{0.4\linewidth} p{0.4\linewidth}}
% 		\hline
% 		John1 Doe1 & Barcelona, Spain \\ \hline
% 		John2 Doe2 & Barcelona, Spain \\ \hline
% 		John3 Doe3 & Barcelona, Spain \\ \hline
% 		John4 Doe4 & Barcelona, Spain \\ \hline
% 		John5 Doe5 & Barcelona, Spain \\ \hline
% 		John6 Doe6 & Barcelona, Spain \\ \hline
% 		John7 Doe7 & Barcelona, Spain \\ \hline
% 		John8 Doe8 & Barcelona, Spain \\ \hline
% 		John9 Doe9 & Barcelona, Spain \\ \hline
% 		John10 Doe10 & Barcelona, Spain \\ \hline
% 		John11 Doe11 & Barcelona, Spain \\ \hline
% 		John12 Doe12 & Barcelona, Spain \\ \hline
% 		John13 Doe13 & Barcelona, Spain \\ \hline
% 		John14 Doe14 & Barcelona, Spain \\ \hline
% 		John15 Doe15 & Barcelona, Spain \\ \hline
% 		John16 Doe16 & Barcelona, Spain \\ \hline
% 		John17 Doe17 & Barcelona, Spain \\ \hline
% 		John18 Doe18 & Barcelona, Spain \\ \hline
% 		John19 Doe19 & Barcelona, Spain \\ \hline
% 		John20 Doe20 & Barcelona, Spain \\ \hline
% 		John21 Doe21 & Barcelona, Spain \\ \hline
% 		John22 Doe22 & Barcelona, Spain \\ \hline
% 		John23 Doe23 & Barcelona, Spain \\ \hline
% 		John24 Doe24 & Barcelona, Spain \\ \hline
% 		John25 Doe25 & Barcelona, Spain \\ \hline
% 		John26 Doe26 & Barcelona, Spain \\ \hline
% 		John27 Doe27 & Barcelona, Spain \\ \hline
% 		John28 Doe28 & Barcelona, Spain \\ \hline
% 		John29 Doe29 & Barcelona, Spain \\ \hline
% 	\end{longtable}
% \endgroup

%----------------------------------------------------------------------------------------
%	 USEFUL INFORMATION
%----------------------------------------------------------------------------------------

% \chapter{Useful Information}

% \textbf{Talks} will be held at the \textbf{Conference Hall-Auditorium} of PRBB. It is situated on the first floor of the central courtyard and
% has independent access from the rest of the building (through stairs located at the ground floor, main entrance of PRBB).

% \textbf{Coffee breaks and lunches} will be offered in the half-covered terrace in front of the main entrance of the conference hall.

% The \textbf{poster session} will be held on Tuesday and Wednesday night on the \textbf{ground floor} of the PRBB.

% Wi-Fi will be available during the conference. The PRBB also provides access to an eduroam network.

% The \textbf{conference dinner} will be held at the "The best restaurant", at Some Street, 39, Barcelona.

% \section{How to get to the PRBB?}

% The PRBB building overlooks the Ronda del Litoral and is next to the twin towers of the Olympic Village: Torre Mapfre and Arts Hotel. The address is Carrer del Dr. Aiguader, 88, 08003 Barcelona, Spain. and and can be reached by:

% \begin{itemize}
% 	\item \textbf{Subway:} yellow line, L4, station Ciutadella/Vila Ol\'{i}mpica,
% 	\item \textbf{Bus:} lines V21, 14, 36, 41, 45, 59, 71, 92, D20,
% 	\item \textbf{Tram:} line 4, stop Vila Ol\'{i}mpica.
% \end{itemize}

% \newpage

% \includegraphics[width=\linewidth]{images/amcos_map}

% %----------------------------------------------------------------------------------------
% %	 PARTNERS & SPONSORS
% %----------------------------------------------------------------------------------------

% \chapter{Partner Institutions and Sponsors}

% The AMCOS conference is part of the COSMOS project, funded by the European Union’s Horizon 2020 research and innovation programme under the Marie Sk\l{}odowska-Curie grant agreement No 642563.

% \section{Sponsors}

% \begin{center}
% 	\includegraphics[width=0.5\textwidth]{images/logos/Partnerlogos/LancasterHelium.jpg}
% \end{center}

% \vfill

%----------------------------------------------------------------------------------------
%	 CLOSING PAGE
%----------------------------------------------------------------------------------------

% \newpage

% \thispagestyle{empty} % Suppress headers and footers on this page
%\pagecolor{myblue} % Coloured background
~

%----------------------------------------------------------------------------------------

\end{document}
